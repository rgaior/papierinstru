\section*{Introduction}
The  origin of  Ultra  High  Energy Cosmic  Rays  (UHECRs) remains  an
unsolved question.  The  interpretation of the abrupt fall off  at the end of the           observed           spectrum          (E           $\geq$
\unit[5$\cdot10^{19}$])\cite{Fukushima:2013yea} is still being debated. It could result from the extinction of sources if the acceleration at source has reached its maximum energy, but can be interpreted as well as the result of the interaction of UHECR with background light, the GZK effect. The measurement of the composition of UHECR at the highest energies is of essential to disentangle between these two scenarios{ref}.\\ The UHECR are detected through the cascade of particles they induce in  the atmosphere, the Extensive Air Shower or EAS, over a large surface of collection to compensate for a very low flux of primary cosmic ray. For instance, the Pierre Auger Observatory is instrumented over \unit[3000]{km$^2$}. The two main techniques to detect EAS currently implemented are:
\begin{itemize}
\item  the  detection of  the EAS  front at   ground  by  an  array  of  particle  detectors (with  a  spacing  of   \unit[1.5]{km} for the Pierre Auger Observatory).
\item  the observation  of the  fluorescence light  emitted  along the
  shower development by the desexcitation of $N_2$ molecules.
\end{itemize}
One can infer the mass composition through various parameters of the extensive air shower. The position of its maximum of development, $\rm X_{max}$, is measured with the fluorescence technique and provides currently the best information on the mass composition but it is limited by the observation time the fluorescence detectors which can operate only during clear   moonless   night   i.e.    $\sim$   10$\%$ of the time. Other mass sensitive parameters are found with improved data analysis of the surface detector data, like the muon production depth~\cite{augermassmpd} or signal  asymmetry,  but these methods still show some limitations in terms of precision and model dependence. The upgrade undertaken by the Pierre Auger Collaboration aims at the measurement of the mass composition using the information of an additional scintillator allowing for the separation of muonic and electromagnetic components of the shower. An alternative solution would be the measurement of the EAS electromagnetic longitudinal profile with a 100\% duty cycle. This would allow for the estimation of two mass sensitive parameters: $\rm X_{max}$ and the ratio of the muonic and electromagnetic components by combination with the particle detector.
%
%The distribution of the shower energy among the electromagnetic and muonic components is another observable related to the mass composition.

% The combination of a particle measurement at ground with the measurement of the longitudinal profile through the radio component would provide two observables related to the mass composition measurement, enabling the measurement of the primary mass with an almost 100\% duty cycle.
In this perspective, the observation  of radio  waves  from  EAS  is an interesting technique. Firstly  proposed  and implemented in the 1960's  \cite{jelley65}, the radio detection,  of air shower is now  a well established technique mostly exploited in    the   VHF    band~\cite{huegeradioreview}.     In this band it was    shown in~\cite{augerradio} that the observed radiation is mainly produced by
the acceleration  of the  electrons of the  shower in  the geomagnetic
field, and in a smaller portion by the moving charge excess (also know
as Askaryan radiation).  However,  both of these radiations are beamed
forward in the  Cherenkov cone, which is around $\sim$ 1deg.  in  air, and centered around the shower axis.  The resulting  imprint of the radio signal at ground level is generally observable up to a few hundred meters limiting this
technique to densely instrumented  array therefore to primary energies of cosmic rays of around \unit[$10^{18}]{eV}$. The observation of the radio signal at larger distances would allow for sparser installation of detectors and make this technique also efficient at higher energies. This requires the ability to lower the detection threshold, a difficult task because of the radio frequency noise, including man made noise.  \\In 2008, a beam experiment detected
a signal  at microwave  frequencies (1.5-6 GHz) upon  the passage  of a
shower  of  charged particles  in  an anechoic  chamber~\cite{Gorham}.
This signal was interpreted  as Molecular Bremsstrahlung Radiation (MBR) and
its  intensity,  when extrapolated  to  air  shower  energy, could  be
detected in air shower with rather simple radio-detector systems.  The
MBR is produced by the acceleration of the ionization electrons in the
field  of the  molecules  in  the atmosphere.   The  intensity of  the
radiation is directly related with deposited energy by the shower
along the  atmosphere.  The measurement  of the time evolution  of the
radiation would allow one to determine the longitudinal development of
the shower, and the total received  power to measure the energy of the
primary particle.  Furthermore one expects  the intensity of MBR to be
emitted isotropically  and decrease with  the distance squared  and to
extend   further   than  the   previously   detected  radiation   like
geosynchrotron or Askaryan radiation. Finally, the equipment in this range of frequency has already been well developed for other application and can be found for a rather cheap price.\\ This results and the promising features  of  the MBR needed confirmation and led  to  the  development of  beam  experiments \cite{amy}, \cite{maybe}, as well as in situ experiments aiming at the direct  observation  from   air  showers  \cite{midas},  \cite{crome}, \cite{amber}.  In parallel, improved  calculations of the MBR produced by  air  shower  \cite{imen2016}  have  shown that  the  intensity  is exptected to  be lower by at  least two orders of  magnitude.\\ In this
paper, we  present the developments  of EASIER: Extensive Air Shower Identification with Electron Radiometer, a concept of radio detectors  integrated to the Pierre Auger Observatory  Surface Detector (SD). An EASIER detector is radio antenna combined with an envelope detector  installed on a SD station and is triggered with it. Thanks to the triggered mode, the Radio Frequency Interference (RFI)  and especially the man made noise is filtered out and allow for a reduced detection threshold. Thus this setup has the capability to probe the radio signal from UHECRs at large distances from the shower axis.  After the description of the general experimental setup, we details the calibration of two versions implemeted in the C-band ($\rm [3.4 - 4.2] GHz$), the first one EASIER61 is based on the original  estimation of the MBR and the second, GIGADuck, is an improved version with a better sensitivity. We then describe the  method to simulate  the MBR signal and apply it to the EASIER setups to estimate their expected performances.


% LocalWords:  UHECR
